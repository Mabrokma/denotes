%%
%% 201.tex - lab manual for math 201
%%
%% authors: Malcolm Roberts, Samantha Marion
%%
\documentclass{book}
\usepackage{ualab}

%% see solutions?
%\showsolutionstrue  % yes
\showsolutionsfalse % no

\begin{document}

%%%%%%%%%%%%%%%%%%%%%%%%%%%%%%%%%%%%%%%%%%%%%%%%%%%%%%%%%%%%%%%%%%%%%%
%% title page
%%
\pagestyle{empty}
\pagenumbering{roman}

\mktitle{Math 201}{Differential Equations for Engineers}{an
  \emph{unofficial} lab manual}

%%%%%%%%%%%%%%%%%%%%%%%%%%%%%%%%%%%%%%%%%%%%%%%%%%%%%%%%%%%%%%%%%%%%%%
%% disclaimer and copyright page
%%
\newpage\
\vspace{5in}

This manual was prepared for \emph{Graduates at Alberta Methematics
  Etc.} (GAME) by Malcolm Roberts and Samantha Marion, with
contributions from others.  \vspace{0.5in}

This manual is \emph{not} officially endorsed by the University of
Alberta or the Department of Mathematical and Statistical Sciences at
the University of Alberta.
\vspace{0.5in}

\emph{Use at your own risk.  Errors and omissions are expected.}
\vspace{0.5in}

Copyright 2009, Graduates at Alberta Mathematics Etc.  All rights
reserved.

\newpage
\tableofcontents

%%%%%%%%%%%%%%%%%%%%%%%%%%%%%%%%%%%%%%%%%%%%%%%%%%%%%%%%%%%%%%%%%%%%%%
%% introduction
%%
\newpage
\pagestyle{plain}
\pagenumbering{arabic}

\section*{Introduction}

This manual is meant to provide Math 201 students with short recipes
outlining how many (but not all) problems in Math 201 are solved.  The
recipes are demonstrated by example, and supplementary problems are
provided.

%%%%%%%%%%%%%%%%%%%%%%%%%%%%%%%%%%%%%%%%%%%%%%%%%%%%%%%%%%%%%%%%%%%%%%
%% checking your solution
%%
\section*{Checking your solution}

As in many other branches of mathematics, it is always a good idea to
check your solution.  For differential equations, this means plugging
your solution into the original equation, simplifying both sides, and
making sure the simplified equation is satisfied.

We will do this a few times throughout this manual, but, the reader
should always try to do this themselves.

This becomes easier with practise, and will help your understanding of
what's going on!

\begin{example*}
  Determine if $y(x) = \sqrt{3-2\sin x}$ is a solution to the
  differential equation
  \begin{equation*}
    y \dd{y}{x} = \sin(x).
  \end{equation*}
\end{example*}

\begin{solution}
  First, we consider the LHS of the DE.  Subsituting the proposed
  solution into the LHS, we obtain
  \begin{align*}
    y \dd{y}{x} &= \sqrt{3-2\sin x}\ \frac{d}{dx} \sqrt{3-2\sin x} \\
                &= \sqrt{3-2\sin x}\ \frac{-\cos x}{\sqrt{3-2\sin x}} \\
                &= - \cos x.
  \end{align*}
  Next, we consider the RHS of the DE.  The RHS is $\sin(x)$.

  The LHS and RHS do not match ($-\cos x \neq \sin x$), and therefore
  the proposed solution is \emph{not} a solution to the DE.
\end{solution}


%%%%%%%%%%%%%%%%%%%%%%%%%%%%%%%%%%%%%%%%%%%%%%%%%%%%%%%%%%%%%%%%%%%%%%
%% first order
%%
\newpage
\chapter{First order DEs}

% XXX: intro and quick note about how first order DEs may fit into
% more than one category


%%%%%%%%%%%%%%%%%%%%%%%%%%%%%%%%%%%%%%%%%%%%%%%%%%%%%%%%%%%%%%%%%%%%%%
%% separable equations
%%
\newpage
\section{Separable equations}

Separable equations are differential equations of the form
\begin{equation} \label{eq:separable}
  h(y) \dd{y}{x} = g(x).
\end{equation}
They are called \emph{separable} because it is possible to move
everything that depends on $x$ to one side of the equation, and
everything to do with $y$ to the other.

To find the general solution to separable equations:
\begin{enumerate*}
\item \emph{Separate}: move everything to do with $x$ to the left, and
  everything to do with $y$ to the right.
\item \emph{Integrate}: integrate both sides.
\item \emph{Apply conditions}: apply any extra conditions, if
  supplied.
\item \emph{Isolate}: solve the result for $y$ (if possible).
\item \emph{Check}: check the solution for consistency.
\end{enumerate*}

The order of the \emph{apply conditions} and \emph{isolate} steps is
not strict.  That is, you may isolate $y$ first and subsequently apply
any conditions if you prefer.

\begin{heads}
  When integrating, make sure to add a \emph{constant of integration}
  immediately after performing the integration.
\end{heads}

\newpage
\begin{easyexample}
  Find the solution to the differential equation
  \begin{equation}  \label{eq:ex:separable}
    y \dd{y}{x} = \sin(x)
  \end{equation}
  with initial condition $y(0) = 1$.
\end{easyexample}

\begin{solution}
  \solstep{Separate}

  First, we separate by multiplying both sides by $dx$ to obtain
  \begin{equation*}
    y \;dx = \sin(x) \;dx.
  \end{equation*}

  \solstep{Integrate}

  Next, we integrate to obtain
  \begin{equation*}
    \int y \, dy = \int \sin(x) \, dx
    \quad \text{ and hence } \quad
    \frac{y^2}{2} = -\cos(x) + C
  \end{equation*}
  where $C$ is an integration constant that will be determined by the
  initial conditions.

  \solstep{Apply conditions}

  Next, we apply the initial condition.  The initial condition that
  was given is $y(0) = 1$, which means that ``when $x$ is 0, $y$
  should be 1''.  Therefore, subsituting $x=0$ and $y=1$ into our
  solution, we obtain
  \begin{equation*}
    \frac{1^2}{2} = -\cos(0) + C
    \quad \text{ and hence } \quad
    \frac{1}{2} = -1 + C.
  \end{equation*}
  Therefore, we have determined that $C = 3/2$.

  \solstep{Isolate}

  Finally, we isolate $y$ to obtain
  \begin{equation} \label{eq:ex:separable:solution}
    y(x) = \sqrt{3 - 2\cos(x)}.
  \end{equation}
  We have chosen the positive root in order to satisfy the initial
  condition.

  \solstep{Check}

  It's a good idea to check that the solution you obtained is correct.
  This is pretty easy; just plug the solution into the original
  equation.  The solution is given in
  \eqref{eq:ex:separable:solution}, and therefore
  \begin{equation*}
    \dd{y}{x} = \frac{\sin(x)}{\sqrt{3 -2 \cos(x)}}.
  \end{equation*}
  Putting this into equation \eqref{eq:ex:separable} and simplifying
  each side, we obtain
  \begin{gather*}
    y \dd{y}{x} = \sin(x) \\
    \bigl(\sqrt{3 - 2\cos(x)}\bigr) \cdot
    \biggl(\frac{\sin(x)}{\sqrt{3 -2 \cos(x)}}\biggr) = \sin(x) \\
    \sin(x) = \sin(x),
  \end{gather*}
  which is certainly true.  We have verified that the solution we
  obtained is correct.
\end{solution}

\begin{example}
  Find the general solution to
  \begin{equation*}
    x \dd{y}{x} - y = 2 x^2 y.
  \end{equation*}
\end{example}

\begin{solution}
  \solstep{Separate}

  Multiplying by $dx$ and rearranging, we obtain
  \begin{equation*}
    x dy = (2 x^2 y + y) dx.
  \end{equation*}
  Dividing by $x y$, we obtain
  \begin{equation*}
    \frac{dy}{y} = \left(2 x + \frac{1}{x}\right) dx.
  \end{equation*}

  \solstep{Integrate}

  Integrating both sides, we obtain
  \begin{equation*}
    \ln |y| = x^2 + \ln|x| + C.
  \end{equation*}
  This is an implicit solution to the DE.

  \solstep{Isolate}

  The logarithms can be removed from the implicit solution by
  combining the logarithms according to
  \begin{gather*}
    \ln|y| - \ln|x| = x^2 + C \\
    \ln\left| \frac{y}{x} \right| = x^2 + C
  \end{gather*}
  and subsequently exponentiating both sides to obtain
  \begin{equation*}
    \frac{y}{x} = \pm e^{x^2 + C}.
  \end{equation*}
  Finally, multiplying by $x$ and letting $A = e^C$ (or $-e^C$), we
  obtain the general solution
  \begin{equation*}
    y(x) = A x e^{x^2}
  \end{equation*}
  where the sign of $A$ depends on the initial condition (which wasn't
  supplied for this DE).
\end{solution}

\begin{heads}
  You might wonder where the absolute value signs came from and where
  they went.  As to where they came from, recall that for $x>0$ or
  $x<0$ the derivative of $\ln|x| = \frac{1}{x}$ and hence $\int 1/x
  \;dx = \ln|x|$.  As to where they went, strictly speaking we
  obtained two solutions, which were
  \begin{equation*}
    y(x) = x e^{x^2 + C} \quad \text{ and } \quad y(x) = -x e^{x^2 + C}.
  \end{equation*}
  Howeever, both of these are described by
  \begin{equation*}
    y(x) = A x e^{x^2}
  \end{equation*}
  by allowing $A$ to be positive or negative, depending on the initial
  condition.
\end{heads}

% \subsection{Problems}

% \begin{enumerate}

% \item Find the general solution of the ``logistic map'', which is
%   \begin{equation*}
%     \dd{y}{x} = y - y^2.
%   \end{equation*}

% \item Find the general solution of the differential equation
%   \begin{equation*}
%     \ddt{y} = 1 + \frac{1}{y^2}.
%   \end{equation*}

%   \hidesolution{
%     \begin{solution}
%       This is a separable equation.  Separating and integrating, we obtain
%       \begin{equation*}
%         \frac{dy}{1+\frac{1}{y^2}} = dt
%         \implies \int dt = \int \frac{dy}{1+\frac{1}{y^2}}.
%       \end{equation*}
%       But, since
%       \begin{equation*}
%         \frac{1}{1+\frac{1}{y^2}} = \frac{y^2}{y^2+1} = 1 - \frac{1}{1+y^2}
%       \end{equation*}
%       we obtain
%       \begin{equation*}
%         t = \int \biggl[1 - \frac{1}{1+y^2}\biggr] dy = y + \arctan y + C
%       \end{equation*}
%       which implicitly defines $y(t)$.
%     \end{solution}
%   }

% \end{enumerate}

%%%%%%%%%%%%%%%%%%%%%%%%%%%%%%%%%%%%%%%%%%%%%%%%%%%%%%%%%%%%%%%%%%%%%%
%% linear equations
%%
\newpage
\section{Linear equations}

First order linear equations are differential equations of the form
\begin{equation}
  \label{eq:linear}
  \dd{y}{x} + P(x) y = Q(x)
\end{equation}

To find the solution to linear equations:
\begin{enumerate*}
\item \emph{Find the integrating factor}: compute the integrating
  factor $\mu(x)$, which is defined as
  \begin{equation*}
    \mu(x) \doteq \exp \left( \int P(x) \;dx \right).
  \end{equation*}
\item \emph{Multiply by $\mu(x)$}: multiply both sides by $\mu(x)$.
\item \emph{Re-write}: re-write the LHS as one derivative, ie
  \begin{equation*}
    \mu(x) \dd{y}{x} + \mu(x) P(x) y \quad \text{ can be re-written as } \quad \dd{}{x} \bigl( \mu(x) y \bigr).
  \end{equation*}
\item \emph{Integrate}: integrate both sides.
\item \emph{Apply conditions}: apply any extra conditions, if supplied.
\item \emph{Isolate}: solve the result for $y$ (if possible).
\item \emph{Check}: check the solution for consistency.
\end{enumerate*}

This recipe may be shortened by memorising the form of the solution,
which is
\begin{equation*}
  y(x) = \frac{1}{\mu(x)} \left[ \int \mu(x) Q(x) \;dx + C \right].
\end{equation*}

\begin{heads}
  When solving a particular problem, be careful to incorporate the
  sign of $P(x)$ when identifying $P(x)$.
\end{heads}

\begin{heads}
  When computing the integrating factor, it isn't necessary to add a
  constant of integration to $\int P(x)\; dx$, as it would eventually
  divide out.
\end{heads}

\begin{heads}
  You can save yourself a lot of work if you simplify the integrating
  factor as much as you can.  Knowing the various properties of $e^x$
  and $\ln x$ is \emph{very} helpful.
\end{heads}

\newpage
\begin{easyexample}
  Find the solution to the initial value problem
  \begin{equation*}
    (x^2 + 1) \dd{y}{x} + 4xy = 4x, \quad y(0) = 1.
  \end{equation*}
\end{easyexample}

\begin{solution}
  Before we begin, we re-write the DE in order to make it linear.
  Dividing by $x^2 + 1$, we obtain
  \begin{equation*}
    \dd{y}{x} + \frac{4x}{x^2 + 1} y = \frac{4x}{x^2 + 1}.
  \end{equation*}

  \solstep{Find the integrating factor}

  First, we find the integrating factor, which is
  \begin{align*}
    \mu(x) &= \exp\left[ \int P(x) \;dx \right] \\
           &= \exp\left[ \int \frac{4x}{x^2 + 1} \;dx \right] \\
           &= \exp\left[ 2 \ln (x^2 + 1) \right] \\
           &= \exp\left[ \ln (x^2 + 1)^2 \right] \\
           &= (x^2 + 1)^2.
  \end{align*}

  \solstep{Multiply by $\mu(x)$}

  Next, we multiply by $\mu(x)$ to obtain
  \begin{equation*}
    (x^2 + 1)^2 \dd{y}{x} + 4x (x^2 + 1)  y = 4x (x^2 + 1).
  \end{equation*}

  \solstep{Re-write}

  Next, we re-write the LHS as one derivative to obtain
  \begin{equation*}
    \dd{}{x} \left( (x^2 + 1)^2 y \right) = 4x (x^2 + 1).
  \end{equation*}

  \begin{heads}
    This can \emph{always} be done, and is why the integrating factor
    was defined in the way that it was.
  \end{heads}

  \solstep{Integrate}

  Next, we integrate both sides to obtain
  \begin{gather*}
    \int \dd{}{x} \left( (x^2 + 1)^2 y \right) \;dx = \int 4x (x^2 + 1) \; dx
    (x^2 + 1)^2 y = (x^2 + 1)^2 + C.
  \end{gather*}

  \solstep{Apply conditions}

  Next, we apply the initial condition, which was $y(0) = 1$.
  Therefore, subsituting $x=0$ and $y=1$ into our solution, we obtain
  \begin{equation*}
    1 = 1 + C \quad \text{ and hence } \quad C = -1.
  \end{equation*}

  \solstep{Isolate}

  Finally, we isolate $y$ to obtain
  \begin{equation*}
    y(x) = 1 - (x^2 + 1)^{-2}.
  \end{equation*}
\end{solution}

% XXX: add a challenging example


%%%%%%%%%%%%%%%%%%%%%%%%%%%%%%%%%%%%%%%%%%%%%%%%%%%%%%%%%%%%%%%%%%%%%%
%% exact equations
%%
\newpage
\section{Exact equations}

Homogeneous first-order differential equations can be written in the form
\begin{equation}
  \label{eq:exact}
  M(x,y) dx + N(x,y) dy = 0.
\end{equation}
If
\begin{equation*}
  \pp{M(x,y)}{y} = \pp{N(x,y)}{x}
\end{equation*}
then the equation is called \emph{exact}.

To find the solution to exact equations:
\begin{enumerate*}
\item \emph{Check for exactness}: compute the ``cross partials''
  \begin{equation*}
    \pp{M}{y} \quad \text{ and } \quad \pp{N}{x}
  \end{equation*}
  and make sure they match.  Note that $M$ is the thing that
  multiplies $dx$ in \eqref{eq:exact}, and we take its partial
  derivative with respect to $y$ (and similarly for $N$).
\item \emph{Find $F(x,y)$}:
  \begin{enumerate*}
  \item \emph{anti-differentiate}: anti-differentiate the two equations
    \begin{equation*}
      \pp{F}{x} = M \quad \text{ and } \quad \pp{F}{y} = N;
    \end{equation*}
  \item \emph{match terms}: match terms to determine $F(x,y)$.
  \end{enumerate*}
\item \emph{Write implicit solution}: the solution of the original DE is implicity given by
  \begin{equation*}
    F(x,y) = C.
  \end{equation*}
\item \emph{Apply conditions}: apply any extra conditions, if supplied.
\item \emph{Isolate}: solve the result for $y$ (if possible).
\item \emph{Check}: check the solution for consistency.
\end{enumerate*}

\begin{heads}
  It isn't necessary to add integration constants in the
  \emph{anti-differentiate} step, as these are taken care of in the
  \emph{write implicit solution} step.
\end{heads}

\newpage
\begin{easyexample}
  Find the general solution to
  \begin{equation*}
    y \;dx + \bigl( y^2 + x \bigr) \;dy = 0.
  \end{equation*}
\end{easyexample}

\begin{solution}
  \solstep{Check for exactness}

  First, we check to make sure the DE is exact.  We identify $M(x,y) =
  y$ and $N(x,y) = y^2 + x$ and take their cross-partials to obtain
  \begin{equation*}
    \pp{M}{y} = 1 \quad \text{ and } \quad \pp{N}{x} = 1.
  \end{equation*}
  Since the cross-partials match, the equation is exact.

  \solstep{Find $F(x,y)$}

  Next, we find $F(x,y)$ by anti-differentiating the two equations
  \begin{equation*}
    \pp{F}{x} = M \quad \text{ and } \quad \pp{F}{y} = N.
  \end{equation*}
  We obtain
  \begin{align*}
    \pp{F}{x} = y \quad \text{ implies that } \quad F(x,y) &= xy + g(y) \\
    \pp{F}{y} = y^2 + x \quad \text{ implies that } \quad F(x,y) &= \frac{1}{3} y^3 + xy + h(x).
  \end{align*}
  Therefore, matching terms in the two versions of $F(x,y)$ above, we
  conclude that
  \begin{equation*}
    F(x,y) = \frac{1}{3} y^3 + xy.
  \end{equation*}

  \solstep{Write implicit solution}

  Next, we write the (implicit) solution as $F(x,y) = C$ to obtain
  \begin{equation*}
    \frac{1}{3} y^3 + xy = C.
  \end{equation*}

  We weren't given any conditions, and we can't isolate $y$, so this
  is our final solution.
\end{solution}

% XXX: add a challenging example


%%%%%%%%%%%%%%%%%%%%%%%%%%%%%%%%%%%%%%%%%%%%%%%%%%%%%%%%%%%%%%%%%%%%%%
%% transformations
%%
\newpage
\section{Transformations}

Not all first order differential equations are separable, linear, or
exact.  However, sometimes it is possible to make a substitution to
transform a differential equation into something that we already know
how to solve.  This can be a very powerful technique.  Unfortunately,
each type of equation needs its own particular substitution.

The types of equations that can be solved using transformations, and
that you will cover in class, are summarised in the table below.  We
will go through the details of each substitution in subsequent
subsections.

\begin{center}
  \begin{tabular}{cc}
    Form of DE & Substitution \\ \midrule \hspace{2in} & \hspace{2in} \\
    $\dd{y}{x} = f\left( \frac{y}{x} \right)$
      & $v = \frac{y}{x}$ \\ \ \\
    $\dd{y}{x} + P(x) y = Q(x)y^n$
      & $v = y^{1-n}$ \\ \ \\
    $\bigl(a_1 x + b_1 y + c_1\bigr)dx +2 \bigl(a_2 x + b_2 y + c_2\bigr)dy = 0$
      & $x = u+h, \quad y = v+k$ \\ \ \\
    $\dd{y}{x} = f(ax+by)$
      & $v = ax + by$
  \end{tabular}
\end{center}

To solve a differential equation using a substitution, we
\begin{enumerate*}
\item Determine the form of the DE.
\item Substitute.
\item Compute $\dd{y}{x}$ in terms of $\dd{v}{x}$.
\item Re-write the DE in terms of $v$ and $x$ (no $y$'s should remain).
\item Solve the resulting DE.
\end{enumerate*}

\newpage
\subsection{(Homogeneous) equations of the form $y' = f(y/x)$}

Equations of form
\begin{equation}
  \label{eq:homogeneous}
  \dd{y}{x} = f\left(\frac{y}{x}\right)
\end{equation}
are transformed into separable equations by using the subsitution
\begin{equation}
  \label{eq:homogeneous_subsitution}
  v = \frac{y}{x}.
\end{equation}
Isolating $y$ in \eqref{eq:homogeneous_subsitution} and
differentiating, we obtain
\begin{equation}
  \label{eq:homogeneous_differential}
  \dd{y}{x} = \dd{}{x} \bigl( x v \bigr) = v + x \dd{v}{x}.
\end{equation}

\begin{easyexample}
  Find the general solution to:
  \begin{equation*}
    \dd{y}{x} = \frac{x}{y}, \quad y(1) = 2.
  \end{equation*}
\end{easyexample}

\begin{solution}
  XXX
\end{solution}

\newpage
\subsection{(Bernoulli) equations of the form $y' + P(x) y = Q(x) y^n$}

% XXX: finish this

$v = y^{1-n}$

% XXX: add an easy example

\newpage
\subsection{Equations of the form $\bigl(a_1 x + b_1 y + c_1\bigr)dx +2 \bigl(a_2 x + b_2 y + c_2\bigr)dy = 0$}

% XXX: finish this

$x = u+h, \quad y = v+k$

% XXX: add an easy example

\newpage
\subsection{Equations of the form $y' = f(ax+by)$}

% XXX: finish this

$v = ax + by$

% XXX: add an easy example

\newpage
\subsection{More examples}

% XXX: add challenging homogeneous example

% XXX: add challenging bernoulli example

% XXX: add challenging (a x + b y + c) dx + ... = 0 example

% XXX: add challenging y' = f(ax + by) example

% XXX: add an interesting example

% XXX: add an interesting example


%%%%%%%%%%%%%%%%%%%%%%%%%%%%%%%%%%%%%%%%%%%%%%%%%%%%%%%%%%%%%%%%%%%%%%
%% second order
%%
\newpage
\chapter{Second-order DEs}

% XXX: intro

\newpage
\section{Homogeneous linear equations}

Linear second order homogeneous differential equations are equations
of the form
\begin{equation}
  \label{eq:second_order_linear_homogeneous}
  a \ddtwo{y}{x} + b \dd{y}{x} + c y = 0
\end{equation}
where $a$, $b$, and $c$ are constants.

To find the general solution to a second order linear homogeneous equation:
\begin{enumerate}
\item \emph{Auxiliary equation}: write the auxiliary equation, which
  is
  \begin{equation*}
    a r^2 + b r + c = 0.
  \end{equation*}
  That is, the auxiliary equation is the DE with $y''$ replaced by
  $r^2$, $y'$ replaced by $r$, and $y$ replaced by $1$.
\item \emph{Roots}: find the roots of the auxiliary equation.  If a
  factorisation isn't obvious, you can find the roots by using the
  quadratic equation, which is
  \begin{equation*}
    r_{1,2} = \frac{-b \pm \sqrt{b^2 - 4ac}}{2a}.
  \end{equation*}
\item \emph{General solution}: write the general solution.  Once you
  have found the roots of the auxiliary equation, there are three
  possibilities:
  \begin{enumerate}
  \item The roots are real and distinct.  In this case, the general
    solution is
    \begin{equation*}
      y(x) = A e^{r_1 x} + B e^{r_2 x}
    \end{equation*}
    where $A$ and $B$ are constants.
  \item The roots are real but repeated.  In this case, the general
    solution is
    \begin{equation*}
      y(x) = A e^{r x} + B x e^{r x}
    \end{equation*}
    where $A$ and $B$ are constants.
  \item The roots are complex.  In this case, the roots always appear
    in the form $r_{1,2} = \alpha \pm \beta i$, and the general
    solution is
    \begin{equation*}
      y(x) = A e^{\alpha x} \cos(\beta x) + B e^{\alpha x} \sin(\beta x)
    \end{equation*}
    where $A$ and $B$ are constants.
  \end{enumerate}
\item \emph{Apply conditions}: apply any extra conditions, if
  supplied.
\item \emph{Check}: check the solution for consistency.
\end{enumerate}

% XXX: add an easy example

% XXX: add an easy example

\end{document}
