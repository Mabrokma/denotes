%%
%% 201.tex - lab manual for math 201
%%
%% authors: Malcolm Roberts, Samantha Marion
%%
\documentclass{article}
\usepackage{ualab}

%% see solutions?
%\showsolutionstrue  % yes
\showsolutionsfalse % no

\begin{document}

%%%%%%%%%%%%%%%%%%%%%%%%%%%%%%%%%%%%%%%%%%%%%%%%%%%%%%%%%%%%%%%%%%%%%%
%% title page
%%
\pagestyle{empty}
\pagenumbering{roman}

\mktitle{Math 201}{Differential Equations for Engineers}{an
  \emph{unofficial} lab manual}

%%%%%%%%%%%%%%%%%%%%%%%%%%%%%%%%%%%%%%%%%%%%%%%%%%%%%%%%%%%%%%%%%%%%%%
%% disclaimer and copyright page
%%
\newpage\
\pagestyle{empty}
\vspace{5in}

This manual was prepared for \emph{Graduates at Alberta Methematics
  Etc.} (GAME) by Malcolm Roberts and Samantha Marion, with
contributions from others.  \vspace{0.5in}

This manual is \emph{not} officially endorsed by the University of
Alberta or the Department of Mathematical and Statistical Sciences at
the University of Alberta.
\vspace{0.5in}

\emph{Use at your own risk.  Errors and omissions are expected.}
\vspace{0.5in}

Copyright 2009, Graduates at Alberta Mathematics Etc.  All rights
reserved.

%%%%%%%%%%%%%%%%%%%%%%%%%%%%%%%%%%%%%%%%%%%%%%%%%%%%%%%%%%%%%%%%%%%%%%
%% introduction
%%
\pagestyle{plain}
\pagenumbering{arabic}
\newpage

\section*{Introduction}

This manual is meant to provide Math 201 students with short recipes
outlining how many (but not all) problems in Math 201 are solved.  The
recipes are demonstrated by example, and supplementary problems are
provided.

\section*{Checking your solution}

As in many other branches of mathematics, it is always a good idea to
check your solution.  For differential equations, this means plugging
your solution into the original equation, simplifying both sides, and
making sure the simplified equation is satisfied.

We will do this a few times throughout this manual, but, the reader
should always try to do this themselves.

This becomes easier with practise, and will help your understanding of
what's going on!

%%%%%%%%%%%%%%%%%%%%%%%%%%%%%%%%%%%%%%%%%%%%%%%%%%%%%%%%%%%%%%%%%%%%%%
%% separable equations
%%
\newpage

\section{Separable equations}

Separable equations are differential equations of the form
\begin{equation} \label{eq:separable}
  h(y) \dd{y}{x} = g(x).
\end{equation}
They are called \emph{separable} because it is possible to move
everything that depends on $x$ to one side of the equation, and
everything to do with $y$ to the other.

To solve separable equations:
\begin{enumerate*}
\item \emph{separate}: move everything to do with $x$ to the left, and
  everything to do with $y$ to the right;
\item \emph{integrate}: integrate both sides;
\item \emph{apply conditions}: apply any extra conditions, if
  supplied;
\item \emph{isolate}: solve the result for $y$ (if possible);
\item \emph{check}: check the solution for consistency.
\end{enumerate*}

The order of the \emph{apply conditions} and \emph{isolate} steps is
not strict.  That is, you may isolate $y$ first and subsequently apply
any conditions if you prefer.

\begin{heads}
  When integrating, make sure to add a \emph{constant of integration}
  immediately after performing the integration.
\end{heads}

\begin{example}
  Consider the equation
  \begin{equation}  \label{eq:ex:separable}
    y \dd{y}{x} = \sin(x)
  \end{equation}
  with initial condition $y(0) = 1$.
\end{example}

\begin{solution}
  \solstep{Separate}

  First, we \emph{separate} by
  multiplying both sides by $dx$ to obtain
  \begin{equation*}
    y \;dx = \sin(x) \;dx.
  \end{equation*}

  \solstep{Integrate}

  Next, we \emph{integrate} to obtain
  \begin{equation*}
    \int y \, dy = \int \sin(x) \, dx
    \quad \text{ and hence } \quad
    \frac{y^2}{2} = -\cos(x) + C
  \end{equation*}
  where $C$ is an integration constant that will be determined by the
  initial conditions.

  \solstep{Apply conditions}

  Next, we \emph{apply the initial condition}.  The initial condition
  that was given is $y(0) = 1$, which means that ``when $x$ is 0, $y$
  should be 1''.  Therefore, subsituting $x=0$ and $y=1$ into our
  solution, we obtain
  \begin{equation*}
    \frac{1^2}{2} = -\cos(0) + C
    \quad \text{ and hence } \quad
    \frac{1}{2} = -1 + C.
  \end{equation*}
  Therefore, we have determined that $C = 3/2$.

  \solstep{Isolate}

  Finally, we \emph{isolate} for $y$ to obtain
  \begin{equation} \label{eq:ex:separable:solution}
    y(x) = \sqrt{3 - 2\cos(x)}.
  \end{equation}
  We have chosen the positive root in order to satisfy the initial
  condition.

  \solstep{Check}

  It's a good idea to check that the solution you obtained is correct.
  This is pretty easy; just plug the solution into the original
  equation.  The solution is given in
  \eqref{eq:ex:separable:solution}, and therefore
  \begin{equation*}
    \dd{y}{x} = \frac{\sin(x)}{\sqrt{3 -2 \cos(x)}}.
  \end{equation*}
  Putting this into equation \eqref{eq:ex:separable} and simplifying
  each side, we obtain
  \begin{gather*}
    y \dd{y}{x} = \sin(x) \\
    \bigl(\sqrt{3 - 2\cos(x)}\bigr) \cdot
    \biggl(\frac{\sin(x)}{\sqrt{3 -2 \cos(x)}}\biggr) = \sin(x) \\
    \sin(x) = \sin(x),
  \end{gather*}
  which is certainly true.  We have verified that the solution we
  obtained is correct.
\end{solution}

\subsection{Problems}

\begin{enumerate}
\item Determine if $y(x) = \sqrt{3-2\sin x}$ solves the differential
  equation
  \begin{equation*}
    y\frac{dy}{dx}=\sin(x).
  \end{equation*}

\item Find the general solution of the ``logistic map'', which is
  \begin{equation*}
    \dd{y}{x} = y - y^2.
  \end{equation*}

\item Find the general solution of the differential equation
  \begin{equation*}
    \ddt{y} = 1 + \frac{1}{y^2}.
  \end{equation*}

  \hidesolution{
    \begin{solution}
      This is a separable equation.  Separating and integrating, we obtain
      \begin{equation*}
        \frac{dy}{1+\frac{1}{y^2}} = dt
        \implies \int dt = \int \frac{dy}{1+\frac{1}{y^2}}.
      \end{equation*}
      But, since
      \begin{equation*}
        \frac{1}{1+\frac{1}{y^2}} = \frac{y^2}{y^2+1} = 1 - \frac{1}{1+y^2}
      \end{equation*}
      we obtain
      \begin{equation*}
        t = \int \biggl[1 - \frac{1}{1+y^2}\biggr] dy = y + \arctan y + C
      \end{equation*}
      which implicitly defines $y(t)$.
    \end{solution}
  }

\end{enumerate}

\end{document}
